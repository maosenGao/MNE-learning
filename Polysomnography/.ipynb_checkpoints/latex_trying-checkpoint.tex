    \documentclass{article}
    \usepackage{CJK}
     
    %要运行该模板,LaTex需要安装CJK库以支持汉字.
    %字体大小为12像素,文档类型为article
    %如果你要写论文,就用report代替article
    %所有LaTex文档开头必须使用这句话
    %使用支持汉字的CJK包
     
    %开始CJK环境,只有在这句话之后,你才能使用汉字
    %另外,如果在Linux下,请将文件的编码格式设置成GBK
    %否则会显示乱码
    \begin{CJK*}{GBK}{song}
    %这是文章的标题
    \title{LaTex 常用模板}
    %这是文章的作者
    \author{Kevin}
    %这是文章的时间
    %如果没有这行将显示当前时间
    %如果不想显示时间则使用 \date{}
    \date{2008/10/12}
    %以上部分叫做"导言区",下面才开始写正文
    \begin{document}
    %先插入标题
    \maketitle     %主要的作用是用于生成标题的作用 content contain \title \author \date
    %再插入目录
    \tableofcontents   %主要的作用适用于生成目录的作用
    \section{LaTex 简介}
    LaTex是一个宏包,目的是使作者能够利用一个
    预先定义好的专业页面设置,
    从而得以高质量的排版和打印他们的作品.
    %第二段使用黑体,上面的一个空行表示另起一段
    \CJKfamily{hei}LaTex 将空格和制表符视为相同的距离.
    多个连续的空白字符 等同为一个空白字符
    \section{LaTex源文件}
    %在第二段我们使用隶书
    \CJKfamily{li}LaTex 源文件格式为普通的ASCII文件,
    你可以使用任何文本编辑器来创建.
    LaTex源文件不仅包括你要排版的文本, 还包括LaTex
    所能识别的,如何排版这些文本的命令.
    \section{结论}
    %在结论部分我们使用仿宋体
    \CJKfamily{fs}LaTeX, 我看行!
    \end{CJK*}
    \end{document}

这个是超级简洁版本的

    \documentclass{article}
    \usepackage{CJK}
     
    \begin{CJK*}{GBK}{song}
    \begin{document}
     
    \section{LaTex 简介}
    输入想要写作的内容
    \end{CJK*}
    \end{document}
————————————————
版权声明:本文为CSDN博主「LJ_Huang」的原创文章,遵循 CC 4.0 BY-SA 版权协议,转载请附上原文出处链接及本声明。
原文链接:https://blog.csdn.net/huang_shao1/article/details/82180825